\documentclass{article}

\usepackage{amsthm}
\usepackage{amsfonts}
\usepackage{amsmath}
\usepackage{amssymb}
\usepackage{fullpage}
\usepackage[usenames]{color}
\usepackage{hyperref}
  \hypersetup{
    colorlinks = true,
    urlcolor = blue,       % color of external links using \href
    linkcolor= blue,       % color of internal links 
    citecolor= blue,       % color of links to bibliography
    filecolor= blue,        % color of file links
    }
    
\usepackage{listings}

\definecolor{dkgreen}{rgb}{0,0.6,0}
\definecolor{gray}{rgb}{0.5,0.5,0.5}
\definecolor{mauve}{rgb}{0.58,0,0.82}

\lstset{frame=tb,
  language=haskell,
  aboveskip=3mm,
  belowskip=3mm,
  showstringspaces=false,
  columns=flexible,
  basicstyle={\small\ttfamily},
  numbers=none,
  numberstyle=\tiny\color{gray},
  keywordstyle=\color{blue},
  commentstyle=\color{dkgreen},
  stringstyle=\color{mauve},
  breaklines=true,
  breakatwhitespace=true,
  tabsize=3
}

\theoremstyle{theorem} 
   \newtheorem{theorem}{Theorem}[section]
   \newtheorem{corollary}[theorem]{Corollary}
   \newtheorem{lemma}[theorem]{Lemma}
   \newtheorem{proposition}[theorem]{Proposition}
\theoremstyle{definition}
   \newtheorem{definition}[theorem]{Definition}
   \newtheorem{example}[theorem]{Example}
\theoremstyle{remark}    
  \newtheorem{remark}[theorem]{Remark}


\title{CPSC-354 Report}
\author{Your Name  \\ Chapman University}

\date{\today}

\begin{document}

\maketitle

\begin{abstract}
Short  summary of purpose and content.  
\end{abstract}

\tableofcontents

\section{Introduction}\label{intro}

\section{Homework}\label{homework}

This section contains solutions to homework. 

\subsection{Week 1}

\subsection{Week 2}

\ldots

\section{Project}

Introductory remarks. The following structure should be suitable for most programming projects. 

\subsection{Specification}
\subsection{Prototype}
\subsection{Documentation}
\subsection{Critical Appraisal}

\section{Conclusions}\label{conclusions}

(approx 400 words) A critical reflection on the content of the course. Step back from the technical details. How does the course fit into the wider world of programming languages and software engineering?

\begin{thebibliography}{99}
\bibitem[PL]{PL} \href{https://github.com/alexhkurz/programming-languages-2022/blob/main/README.md}{Programming Languages 2022}, Chapman University, 2022.
\end{thebibliography}

\end{document}
